\section{Introduction}
% no \IEEEPARstart
% This demo file is intended to serve as a ``starter file''
% for IEEE conference papers produced under \LaTeX\ using
% IEEEtran.cls version 1.8 and later.
% You must have at least 2 lines in the paragraph with the drop letter
% (should never be an issue)
% I wish you the best of success.

Here is some sample LaTeX notation. By associativity, if $\zeta$ is combinatorially closed then $\delta = \Psi$. Since $${S^{(F)}} \left( 2, \dots,-\mathbf{{i}} \right) \to \frac{-\infty^{-6}}{\overline{\alpha}},$$ $l < \cos \left( \hat{\xi} \cup P \right)$. Thus every functor is Green and hyper-unconditionally stable. Obviously, every injective homeomorphism is embedded and Clifford. Because $\mathcal{{A}} > S$, $\tilde{i}$ is not dominated by $b$. Thus ${T_{t}} > | A |$.

\subsection{Subsection Heading Here}
Subsection text here. Let's show some more LaTeX: Obviously, ${W_{\Xi}}$ is composite. Trivially, there exists an ultra-convex and arithmetic independent, multiply associative equation. So $\infty^{1} > \overline{0}$. It is easy to see that if ${v^{(W)}}$ is not isomorphic to $\mathfrak{{l}}$ then there exists a reversible and integral convex, bounded, hyper-Lobachevsky point. One can easily see that $\hat{\mathscr{{Q}}} \le 0$. Now if $\bar{\mathbf{{w}}} > h' ( \alpha )$ then ${z_{\sigma,T}} = \nu$. Clearly, if $\| Q \| \sim \emptyset$ then every dependent graph is pseudo-compactly parabolic, complex, quasi-measurable and parabolic. This completes the proof.

\subsubsection{Subsubsection Heading Here}
Subsubsection text here. This is how you can cite other articles. Just type \verb|\cite{DOI}| where DOI is a Digital Object Identifier. For example cite \href{http://dx.doi.org/10.1109/INFCOM.2001.916703}{this article published in IEEE INFOCOM 2001} \cite{Aad_Castelluccia_2001}